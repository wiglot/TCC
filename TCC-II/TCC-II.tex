


\documentclass[pnumromarab,normaltoc]{abnt}			%Numera��o de acordo com UFPR

% Utilize a op��o normalfigtabnum para numerar as figuras e tabelas por cap�tulo

\usepackage[brazil]{babel}
\usepackage[latin1]{inputenc}
\usepackage{abnt-alf}
\usepackage{graphicx}														%Package para figuras

% % % % % % % % % % % % % % % % % % % % % % % % % % % % % % % % % % % % %
\usepackage{float}
\usepackage{amsthm}
\usepackage{amsmath}
\usepackage{setspace}

\floatstyle{ruled}
\newfloat{algorithm}{tbp}{loa}[chapter]
\floatname{algorithm}{Algorithm}

%%%%%%%%%%%%%%%%%%%%%%%%%%%%%% User specified LaTeX commands.
\usepackage{tikz}
\usetikzlibrary{snakes,arrows,shapes}
\usepackage{lscape}

% % % % % % % % % % % % % % % % % % % % % % % % % % % % % % % % % 

\makeatletter	%Para que ele entenda o @



\begin{document}
\autor{Wagner de Melo Reck}

\titulo{Problema de dispacho de ordens de servi�os visto como problema de agrupamento}

\orientador{Vin�cius Jacques Garcia}

\comentario{Monografia apresentada para obten��o do Grau de Bacharel em Ci�ncia da Computa��o pela Universidade Federal do Pampa.}

\local{Alegrete}

\data{Junho 2010}


% ELEMENTOS PR�-TEXTUAIS
% Capa - Obrg
\capa
% Lombada
% Folha de rosto - Obrg
\renewcommand{\folhaderosto}%
{%
\begin{titlepage}
\espaco{1.1}
\ABNTifnotempty{\ABNTautordata}%
  {%
  \begin{center}
    \autorformat\ABNTautordata
  \end{center}
  }
\vfill\vfill\vfill
\ABNTifnotempty{\ABNTtitulodata}%
  {%
   \begin{center}
     {\tituloformat\ABNTtitulodata\par}
   \end{center}
  }%
\ABNTifnotempty{\ABNTcomentariodata}%
  {%
   \vspace{.8cm}
   \hspace{8cm}
     \begin{minipage}{8cm}
       \begin{espacosimples}
         {\comentarioformat\ABNTcomentariodata}\par
       \end{espacosimples}
     \end{minipage}
   }
% \begin{center}
\ABNTifnotempty{\ABNTorientadordata}%
  {%
    \vspace{.7cm}
%  \par $\phantom{linha em branco}$
  \hspace{8cm}
     \begin{minipage}{8cm}
       \begin{espacosimples}
% % % % % % % % % % 	     Orientador!  %%%%%%%%
	    Orientador: {\orientadorformat\ABNTorientadordata}\protect\\
	    \vspace{0.7cm}
       \end{espacosimples}
     \end{minipage}

  }
\ABNTifnotempty{\ABNTcoorientadordata}
  {%
%     \vspace{.7cm}
%  \par $\phantom{linha em branco}$
  \hspace{8cm}
     \begin{minipage}{8cm}
       \begin{espacosimples}
% % % % % % % % % % 	     Orientador!  %%%%%%%%
	    Co-orientador: {\coorientadorformat\ABNTcoorientadordata}\protect\\
	    \vspace{0.7cm}
       \end{espacosimples}
     \end{minipage}

  }
% \end{center}
\vfill
\begin{center}
\begin{espacosimples}
  \setlength{\parskip}{.3cm}
  \ABNTifnotempty{\ABNTinstituicaodata}%
    {%
     \setlength{\parskip}{0cm}
     {\instituicaoformat\ABNTinstituicaodata\par}
     \setlength{\parskip}{.3cm}\par

    }
\end{espacosimples}
\end{center}
\vfill\vfill
\begin{center}
  \ABNTifnotempty{\ABNTlocaldata}
      {{\localformat\ABNTlocaldata}\par}
    \ABNTifnotempty{\ABNTdatadata}
      {{\dataformat\ABNTdatadata}}

\end{center}
\end{titlepage}
}% end of \folhaderosto
\folhaderosto
\addtocounter{page}{1} %Come�a a contar aqui :)
% Errata
% Folha de aprova��o - Obrg
% \begin{titlepage}
% \newgeometry{margin=1cm}
\begin{center}
\vspace{-3.5cm}
\hspace{-3.5cm}
\includegraphics[width=\paperwidth,height=0.9\paperheight,
keepaspectratio]{./images/ficha_aprovacao.png}%
% \restoregeometry{}
%  \includegraphics[scale=0.32]{./images/ficha_aprovacao.png}
%  % ficha_aprovacao.png: 2123x3192 pixel, 96dpi, 56.16x84.44 cm, bb=0 0 1592 2394
\end{center}
% \end{titlepage}

% \put(0,0){
% \parbox[b][\paperheight]{\paperwidth}{%
% \vfill
% \centering
% \includegraphics[width=\paperwidth,height=\paperheight,
% keepaspectratio]{./images/ficha_aprovacao.png}%
% \vfill
% }
% }

% % \vspace{0.75cm}
% 
% \begin{center}
% 	\large\ABNTautordata
% \end{center}
% 
% \vspace{1.5cm}
% 
% \begin{center}
% 	\large\ABNTtitulodata
% \end{center}
% 
% \vspace{1cm}
% 
% \hspace{8cm}
% \begin{minipage}{8cm}
% \begin{espacosimples}
% Monografia apresentada ao Curso de Gradua��o em
% Ci�ncia da Computa��o da Universidade
% Federal do Pampa, como requisito
% parcial para obten��o do T�tulo de
% Bacharel em Ci�ncia da Computa��o.
% \vspace {0.7cm}
% \\�rea de concentra��o: Ci�ncias Exatas e da Terra
% 
% \end{espacosimples}
% \end{minipage}
% 
% \vspace {1cm}
% \begin{center}
% Disserta��o defendida e aprovada em: 12 de Julho de 2010.\\
% Banca examinadora:
% \end{center}
% \vfill
% 
% \setlength{\ABNTsignthickness}{0.4pt}
% \setlength{\ABNTsignskip}{2cm}
% 
% \vspace{-0.5cm}
% \assinatura{Prof. Dr. Vin�cius Jacques Garcia\\Orientador\\UNIPAMPA - Campus Alegrete}
% 
% \vspace{-0.5cm}
% \assinatura{Prof. Msc.. Marcelo Cezar Pinto\\UNIPAMPA - Campus Alegrete}
% 
% \vspace{-0.5cm}
% \assinatura{Prof. Dr. Cleo  Zanella Billa\\UNIPAMPA - Campus Alegrete}
% 
% 
% \end{titlepage}
% Dedicat�ria(s)
\pretextualchapter{Dedicat�ria}

\vspace{12cm}
\hspace{.3\textwidth}
\begin{minipage}{.6\textwidth}
	\par A quem eu dedico,
	\par $\phantom{linha em branco}$
	\par pra minha m�e, pro meu pai, pra minha tia cr�udia
\end{minipage}

\newpage

% ******** AGRADECIMENTOS *********
% *********** OPCIONAL ************
\pretextualchapter{Agradecimentos}

\vspace{12cm}
\hspace{.3\textwidth}
\begin{minipage}{.6\textwidth}
	\par A todos que, direta ou indiretamente, contribu�ram para a realiza��o e divulga��o deste trabalho.
\end{minipage}

% *********** EP�GRAFE ************
% *********** OPCIONAL ************
% \pretextualchapter{Ep�grafe}
%
% \vfill
% \hspace{.3\textwidth}
% \begin{minipage}{.6\textwidth}
% 	\par Uma pequena prosa que esteja relacionada com o trabalho.
% 	\par Obs.: A forma como uma prosa � escrita � importante.
% \end{minipage}
% Agradecimentos
% Ep�grafe
% Resumo em l�ngua vern�cula - Obrg
% Resumo em l�ngua estrangeira - Obrg
% Lista de ilustra��es
\listadefiguras
% Lista de tabelas
\listadetabelas
% Lista de abreviaturas e siglas
% Lista de s�mbolos
% Sum�rio - Obrg
\sumario
 
% ELEMENTOS TEXTUAIS
% Introdu��o Obrg
\chapter*{Introdu��o}
Em alguns setores de empresas, para efetuar o atendimento a um cliente, elas
devem deslocar um funcion�rio (ou uma equipe deles) da sua base de opera��es at�
a casa do cliente para executar o servi�o.
Isso ocorre principalmente em empresas que prestam algum tipo de servi�o de
atendimento domiciliar aos clientes, como por exemplo empresas de entrega de
correspond�ncia, provedores de internet no que diz respeito a suporte aos
clientes e companhias de distribui��o de energia el�trica tamb�m referente aos
suporte a clientes.
% Esse ultimo cen�rio que esse trabalho toma emprestado para os estudos de caso.

Esses tipos de atendimentos aos clientes s�o chamados de ordens de servi�o (OS),
 podendo ter outros nomes conforme a �rea de atua��o da empresa, mas sempre com
 o intuito de registrar que a empresa deve fazer um atendimento a um determinado
cliente.
Cada OS contem alguns dados b�sicos como localiza��o do atendimento e tempo
previsto de atendimento. Trataremos aqui a localiza��o como um ponto no espa�o
$\Re^{2}$ e o tempo como a demanda de tempo que a OS levar� para ser atendida.

Como medida de deslocamento apenas consideraremos apenas a dist�ncia euclidiana
entre os pontos, pois algumas inst�ncias reais se fossem representadas com a
dist�ncia real que a equipe percorreria, iria demandar um mapa completo da
localidade e m�os das ruas, o que nem sempre � dispon�vel.

Nesse cen�rio de estudo tomamos algumas premissas como v�lidas:
\begin{itemize}
 \item Uma OS deve ser atendida por um funcion�rio/equipe;
 \item N�o levamos em considera��o agendamentos de hor�rios, apenas que todas as OS devem ser executas;
 \item Uma OS possui um tempo que a equipe levar� para exuta-la;
 \item Uma equipe consegue executar todas as OS atribuidas a ela no tempo estimado e predefinido de cada uma delas;
 \item O tempo de atendimento de todas as ordens atribuidas a uma equipe n�o deve ultrapassarem sua jornada de trabalho que � pr�viamente estipulada;
 \item N�o ser� considerado o tempo de deslocamento entre os atendimentos das OS.
\end{itemize}

Essas premissas definem o PDOS (Problema de Despacho de Ordens de Servi�o), onde
devemos associar todas OS as equipes de modo que uma OS seja atendida por uma, e
n�o mais que uma, equipe.
Minimizar o deslocamento entre os atendimentos � algo desej�vel para as empresas, pois diminui os custos e o tempo de deslocamento entre os atendimentos pela menor dist�ncia percorrida.

Ao designar um conjunto de OS a uma equipe, deve-se levar em considera��o a
 carga hor�ria de cada mesma e a demanda de tempo que cada OS do conjunto.
 Isso por si s� pode ser visto como um Problema da Mochila, o que j� n�o � um
problema de solu��o trivial.

Ao atribuir as OS levando em considera��o apenas a carga hor�ria da equipe e o tempo demandado pelas OS pode nos levar a solu��es como as apresentadas na figura \ref{FigDemandaCapacidade}, onde temos 3 equipes atendendo um total de 15 atendimentos.
\begin{figure}
 \begin{center}
  \includegraphics[scale=0.5]{./images/DemandasCapacidade.png}
 % DemandasCapacidade.png: 455x299 pixel, 96dpi, 12.04x7.91 cm, bb=0 0 341 224
  \end{center}
\caption{\label{FigDemandaCapacidade}Aloca��o de OS apenas pelas Demandas e
Capacidades }
\end{figure}
� poss�vel perceber que os atendimentos de cada equipe encontram-se
dispersos, necessitando um maior deslcomento das equipes de um atendimento a
outro do que se os atendimentos estivessem agrupados mais pr�ximos.

Um problema que engloba essas caracter�sticas apresentadas � o Problema de
Agrupamento Capacitado (CCP - Capacitated Clustering Problem),
 cl�ssico na literatura e tamb�m conhecido como Problema das P-Medianas Capacitado (CPMP).
 Na figura \ref{FigCCPIntro} temos um exemplo de como as ordens de servi�o poderiam estar associadas �s equipes.

� poss�vel perceber que ainda assim n�o foi poss�vel atribuir as OS de modo que todas OS de de uma equipe fique pr�ximas de uma OS escolhida como centro.
 Isso � devido as restri��es de capacidade das equipes e demandas das OS.

\begin{figure}
 \begin{center}
  \includegraphics[scale=0.5]{./images/CCPIntro.png}
 % DemandasCapacidade.png: 455x299 pixel, 96dpi, 12.04x7.91 cm, bb=0 0 341 224
  \end{center}
\caption{\label{FigCCPIntro} Poss�vel atribui��o das Mesmas OS segundo CCP }
\end{figure}

As inst�ncias in�ditas desse problema foram retiradas de ordens de atendimento
de um concession�ria de distribui��o de energia el�trica e tiveram que ter a
capacidade de atendimento normalisadas para ser poss�vel gerar solu��es v�lidas
do problema. Esse fato ocorre pois para se encontrar uma solu��o � necess�rio
existir uma folga entre das demandas e a capacidade. A frente no trabalho ser�
apresentado um estudo sobre essas folgas.

Nesse trabalho n�o trataremos da gera��o de rotas entre os atendimentos, mas
esse problema j� foi tratado na literatura como o problema do caixeiro viajante
agrupado, onde desejamos entrar os menor ciclo Hamiltoniano, sendo que os pontos
dos agrupamentos devem ser visitados em sequ�ncia \cite{jeanCTSP-Genetic}.

 %Inclui arquivo introducao.tex
% Desenvolvimento - Obrg
\chapter{Problema de despacho de ordens de servi�o (PDOS)}

Para algumas empresas, principalmente quando necessitam atender a
um grande n�mero de clientes em um centro urbano onde � necess�rio
o deslocamento de uma equipe da empresa at� um determinado cliente,
� uma atividade que se n�o for bem planejada pode levar a um ac�mulo
de atendimentos n�o executados e posteriormente ao n�o atendimento
de alguma solicita��o. Como tendencia natural das empresas, as mesmas
tendem a aumentar a carta de clientes e com isso necessitam atender
a mais solicita��es no mesmo tempo que atendiam as anteriores.

Para resolver isso, tendo em vista que o tempo de executar uma atividade
n�o pode ser diminu�do com o passar do tempo mantendo-se em uma m�dia
para cada tipo de atividade, a empresa ou deve contratar mais mais
equipes ou organizar as solicita��es em grupos para minimizando o
tempo de deslocamento entre uma atividade e outra. A primeira � uma
solu��o que traz custos para empresa e necessidade de pessoal para
gerenciamento das equipes extras, as quais podem ter de ser utilizadas
apenas em per�odos onde houve demanda de atendimento, mas no restando
do tempo passam as ser sub-utilizada por as equipes anteriores atenderem
todas as demandas dos clientes.

A solu��o de agrupar as ordens de servi�o de modo a minimizar o deslocamento
de um atendimento at� um outro dentro do mesmo grupo atendido pela
mesma equipe, torna-se melhor que a anterior pois n�o h� a necessidade
de contratar uma equipe nova para dar conta de alguns poucos atendimentos.
Essa solu��o n�o evita o contratamento de novas equipes no caso de
um aumento muito grande na demanda de servi�os, mas diminui o n�mero
total de novas contrata��es uma vez que o os n�mero total de atendimento
atendidos por cada equipe aumenta pela diminui��o do tempo perdido
no deslocamento entre os atendimentos.

Na pr�xima sec��o, ser� apresentado o problema de agrupamento o qual
� proposto para representar o PDOS e ser� explicado mais a frente.


\section{Problema de Agrupamento Capacitado}

O problema de agrupamento capacitado consiste em dado um conjunto
$I$ de $n$ indiv�duos com suas respectivas demandas $d_{i}\: i\in I$,
� desejado agrupar os $n$ pontos em $p$ agrupamentos, onde a soma
das demandas em cada agrupamento deve ser menor que a capacidade total
do agrupamento $Q_{j},\:\forall j\in P$, sendo $P$ o conjunto de
agrupamentos. Dos pontos pertencentes a cada agrupamento, � escolhida
uma mediana de modo a minimizar a dist�ncia de todos os pontos at�
ela. O problema possu� a seguinte formula��o matem�tica \cite{ahmadi2004density}:

\begin{equation}
Min\;\sum_{i\in P}\sum_{j\in I}c_{ij}x_{ij}\label{eq:obj_CCP}\end{equation}


Sujeito �:

\begin{equation}
\sum_{j\in p}x_{ij}=1,\;\forall i\in I\label{eq:1Cluster_CCP}\end{equation}


\begin{equation}
\sum_{j\in P}y_{j}=p\label{eq:numClusters_CCP}\end{equation}


\begin{equation}
x_{ij}\leq y_{j,\;}\forall i\in I,\:\forall j\in P\label{eq:mustAssign_CCP}\end{equation}


\begin{equation}
\sum_{i\in I}q_{i}x_{ij}\leq Q_{j},\;\forall j\in P\label{eq:capacity_CCP}\end{equation}


\begin{equation}
x_{ij},\: y_{j}\in\{0,1\},\;\forall i\in I,\;\forall j\in P\label{eq:sets_CCP}\end{equation}


Em (\ref{eq:obj_CCP}), temos a fun��o objetivo que minimiza a dissimilaridades
entre os componentes de cada agrupamento. Utilizamos a dist�ncia como
medida da similaridade entre dois pontos, tendo assim a medida $c_{ij}$
do ponto $i$ at� o ponto $j$ (mediana do agrupamento), considerando
que o ponto $i$ est� associado ao agrupamento $j$ pela vari�vel
bin�ria $x_{ij}$, sendo $x_{ij}=1$ caso o ponto $i$ esteja associado
ao agrupamento $j$ e $0$ caso contr�rio. A restri��o (\ref{eq:1Cluster_CCP})
indica que cada ponto deve estar associado a um �nico agrupamento
e (\ref{eq:mustAssign_CCP}) imp�e que todos os pontos deve ser atribu�dos
a um agrupamento. J� (\ref{eq:numClusters_CCP}) indica que devem
existir $p$ agrupamentos. Para garantir que a capacidade $Q$ de
um agrupamento n�o ser� ultrapassado, � inserida a restri��o (\ref{eq:capacity_CCP}).
A �ltima restri��o, (\ref{eq:sets_CCP}), especifica as vari�veis
inteiras de decis�o. I � o conjunto de indiv�duos.

Na figura \ref{fig:Exemplo-CCP} � apresentado uma poss�vel solu��o
para o CCP.

%
% \begin{figure}[h]
% \begin{centering}
% \includegraphics[scale=0.5]{imagens/cluster_Density_SJC4a}
% \par\end{centering}
% 
% \caption{\label{fig:Exemplo-CCP}Exemplo de uma solu��o do CCP}
% 
% \end{figure}


Quando temos o valor de capacidade dos agrupamentos homog�neo, ou
seja, o valor de $Q_{j}=Q_{i}\:,\forall j\: e\:\forall i\in P$, podemos
dizer que esse � um problema das $p-medianas$ capacitado.

Um outro problema de agrupamento foi pro posto por Negreiro e Palhano\cite{negreiros2006capacitated}
� o CCCP (Capacitated Centred Clustrering Problem - Problema de agrupamento
centrado capacitado) onde os agrupamentos passam a ter seu centro
n�o necessariamente em um ponto (mediana), mas no centroide calculado
entre os pontos pertencentes a cada agrupamento.

s�o propostas 2 varia��es do problema, $p-CCCP$, onde � dado o n�mero
de agrupamentos que se deseja encontrar, e $g-CCCP$ (generic - CCCP)
que se deseja encontrar o menor n�mero de agrupamentos para atender
a todas demandas, sendo a fun��o objetivo em ambas a minimiza��o das
dissimilaridades entre os componentes de cada agrupamento.

Na figura \ref{fig:CCCP} � apresentada a visualiza��o de uma solu��o
para o problema $CCCP$.

%
% \begin{figure}[h]
% \begin{centering}
% \includegraphics{imagens/CCCP}
% \par\end{centering}
% 
% \caption{\label{fig:CCCP}Uma solu��o para CCCP \cite{negreiros2006capacitated}}
% 
% \end{figure}


A primeira, $p-CCCP$, possui o n�mero fixo de agrupamentos que se
deseja encontrar e a segunda varia��o, $g-CCCP$, assume que para
cada novo agrupamento adicionado, � acrescentado um valor de penaliza��o
na fun��o objetivo, levando a minimiza��o de agrupamentos. A defini��o
para o $p-CCCP$ �:

\begin{equation}
min\sum_{i\in I}\sum_{j\in P}\parallel a_{i}-\overline{g_{j}}\parallel^{2}x_{ij}\label{eq:obj_CCCP}\end{equation}


Sujeito �,

\begin{equation}
\sum_{j\in P}x_{ij}=1,\;\forall i\in I,\label{eq:1Cluster_CCCP}\end{equation}


\begin{equation}
\sum_{i\in I}x_{ij}=n_{j},\;\forall j\in P,\label{eq:nPerCluster_CCCP}\end{equation}


\begin{equation}
\sum_{i\in I}a_{i}x_{ij}=n_{j}\overline{g_{j}},\;\forall j\in P,\label{eq:centroid_CCCP}\end{equation}


\begin{equation}
\sum_{i\in I}q_{i}x_{ij}\leq Q_{j},\;\forall j\in P,\label{eq:capacity_CCCP}\end{equation}


\begin{equation}
a_{i}\in\Re^{l},\;\overline{g_{j}}\in\Re^{l},\; n_{j}\in N,\; x_{ij}\in\{0,1\},\;\;\forall i\in I,\;\forall j\in P,\label{eq:sets_CCCP}\end{equation}


onde, $\overline{g_{j}}$ � o centroide do agrupamento $j$, $n_{j}$o
n�mero de indiv�duos associados ao agrupamento $j$, $a_{i}$a posi��o
do indiv�duo $i$ no espa�o $\Re^{l}$, $q_{i}$a demanda do indiv�duo
$i$, $Q_{j}$ a capacidade de cada agrupamento, $I$ o conjunto de
indiv�duos e $P$ o conjunto de agrupamentos.

A fun��o objetivo (\ref{eq:obj_CCCP}) tem como diferencial em rela��o
ao problema cl�ssico $CCP$, o fato das similaridades serem medidas
em rela��o a todos pontos do agrupamento at� seu centroide, o qual
n�o � necessariamente um indiv�duo. A eq. (\ref{eq:1Cluster_CCCP})
indica que um indiv�duo somente pode estar associado a um agrupamento,
e a restri��o (\ref{eq:nPerCluster_CCCP}) associa a $n_{j}$o n�mero
de indiv�duos no agrupamento $j$. Em (\ref{eq:centroid_CCCP}) �
definido todos os centroides dos respectivos agrupamentos e em (\ref{eq:capacity_CCCP})
� colocado que cada agrupamento n�o pode maior demanda dos indiv�duos
que sua pr�pria capacidade. A ultima restri��o (\ref{eq:sets_CCCP})
indica o dom�nio das vari�veis do problema.

No $g-CCCP$ o n�mero de agrupamentos n�o � definido a priori, mas
tem um limitante inferior dado por $\lceil\sum_{i\in I}q_{i}/\sum_{j\in P}Q_{j}\rceil$.
Esse problema a seguinte formula��o:

\begin{equation}
min\;\left(F\sum_{j\in P}z_{j}\right)+\sum_{j\in P}\left(\sum_{i\in I}\parallel a_{i}-\overline{g_{j}}\parallel^{2}x_{ij}\right)\label{eq:obj_g-CCCP}\end{equation}


Sujeito �,

\begin{equation}
\sum_{j\in P}x_{ij}=1,\;\forall i\in I,\label{eq:1Cluster_g-CCCP}\end{equation}


\begin{equation}
\sum_{i\in I}a_{i}x_{ij}=\overline{g_{j}}\left(\sum_{i\in I}x_{ij}\right),\;\forall j\in P,\label{eq:centroid_g-CCCP}\end{equation}


\begin{equation}
\sum_{i\in I}q_{i}x_{ij}\leq Q_{j}z_{j},\;\forall j\in P,\label{eq:capacity_g-CCCP}\end{equation}


\begin{equation}
\overline{g_{j}}\in\Re^{l},\: z_{j},\: x_{ij}\in\{0,1\},\;\forall i\in I,\forall j\in P,\label{eq:sets_g-CCCP}\end{equation}


onde, $z_{j}=1$ se o agrupamento $j$ est� aberto (em uso), $0$
caso contr�rio, $Q_{j}$ a capacidade do agrupamento $j$, $q_{i}$a
demanda o indiv�duo $i$, $\overline{g_{j}}$ o centroide do agrupamento
$j$. $F$ � o custo fixo para abrir um novo agrupamento.

A fun��o objetivo (\ref{eq:obj_g-CCCP}) minimiza as dissimilaridades
entre os integrantes de um mesmo agrupamento al�m do n�mero total
de agrupamentos abertos. A eq. (\ref{eq:1Cluster_g-CCCP}) define
que um indiv�duo somente pode estar associado a um agrupamento, (\ref{eq:centroid_g-CCCP})
encontra o centroide de cada agrupamento e (\ref{eq:capacity_g-CCCP})
garante que a capacidade dos agrupamentos n�o sejam ultrapassados.
As ultimas restri��es (\ref{eq:sets_g-CCCP}) especificam as vari�veis
de decis�o.


\section{Resolu��o do problema de agrupamento}

O CCP, � descrito na literatura como um problema $\mathcal{NP}-completo$\cite{osman1994capacitated},
n�o existindo algoritmos em tempo polinomial para resolver o problema.
Para os m�todos exatos, que apresentam melhores solu��es, eles somente
podem ser utilizados para inst�ncia pequenas (<100 indiv�duos), uma
vez que � um problema de programa��o inteira e o num�ro de variaveis
de decis�o � da ordem de $n^{2}+n$\cite{RePEc:eee:ejores:v:18:y:1984:i:3:p:339-348},
onde $n$ � o n�mero pontos de demanda do problema. Para inst�ncias
muito grandes, o uso de heur�sticas traz solu��es muito boas, algumas
vezes muito pr�ximas aos m�todos exatos. Ser�o apresentadas algumas
heur�sticas para constru��o de uma solu��o inicial do problema de
agrupamento, a qual poder� ser evoluida com algumas t�cnicas:
\begin{description}
\item [{CCP}]~

\begin{description}
\item [{Farthest:}] Proposta por Osman e Christofides em \cite{osman1994capacitated},
tem como base escolher os pontos mais afastados para serem os primeiros
centros
\item [{Density:}] Proposta por Ahmadi e Osman em \cite{ahmadi2004density},
visa construir uma solu��o inicial usando a densidade de pontos para
construir a solu��o inicial.
\end{description}
\item [{CCCP}]~

\begin{description}
\item [{Balanced}] q-trees: que utiliza de arvores q-trees e m�todos como
k-means, Forgy, entre outros
\item [{Unconstrained}] to Constrained: inicia como uma solu��o infact�vel
alterando-a at� conseguir uma solu��o fact�vel. Ambos m�todos para
CCCP apresentados em \cite{negreiros2006capacitated};
\end{description}
\end{description}

\subsection{CCP}

Para o problema de agrupamento capacitado (CCP), apresentamos 2 heur�sticas:


\subsubsection{Farthest}

Osman e Christofides apresentaram a heuristica construtiva com base
no afastamento inicial dos primeiros centros (mais afastados entre
s�) e ap�s um rec�lculo dos centros de cada agrupamento. Essa � uma
heuristica muito r�pida mas que pode n�o encontrar solu��es fact�ves
para o caso de a soma de todas demandas seja muito pr�xima das capacidades
de todos agrupamentos. A heur�stica pode ser apresentada da seguinte
forma:
\begin{itemize}
\item Passo 1 - encontre os pontos $(i,j)$ mais afastados entre s�, ent�o
$C={i,j}$

\begin{itemize}
\item Se n�mero desejado de agrupamentos (p) = 2, v� para o passo 3.
\end{itemize}
\item Passo 2 - enquanto |C| < n�mero desejado de agrupamento, fa�a

\begin{itemize}
\item Encontre um cento $k\in O-C$, de modo que:
\end{itemize}
\[
\prod_{j\in C}d_{kj}=\max_{_{k\in O-C}}\;\;\prod_{j\in C}d_{kj}\]

\begin{itemize}
\item Ent�o fa�a $C=C\cup k$;
\end{itemize}
\item Passo 3 - Para cada consumidor, encontre a dist�ncia at� o centro
mais pr�ximo. Organize estas dist�ncias em ordem crescentedo. Atribua
os consumidores na ordem dessas dist�ncias aos centros correspondentes,
se a capacidade permitir, caso contr�rio, atribua para o centro mais
pr�ximo dispon�vel.
\item Passo 4 - Recalcule os centros dos agrupamentos de modo a minimizar
as dist�ncia entre os pontos e o novo centro.
\end{itemize}
Essa heuristica tem como objetivo a constru��o r�pida de uma solu��o
e para se obter melhores solu��es deve-se empregar heuristicas de
busca na visinhan�a da solu��o obtida com essa heuristica construtiva.
Em \cite{osman1994capacitated} � apresentado um metodo hibido que
combina Busca tabu e simulated annealing.


\subsubsection{Density}

Para resolver um problema de agrupamento uma abordagem poss�vel para
escolha dos centros iniciais � a escolha dos centros que possuam uma
maior densidade de pontos, assim a dist�ncia entre esse centro candidato
e os poss�veis pontos componentes ser� minimizada. Essa abordagem
tem carater guloso e � miope por escoher os pontos candidatos a centros
vendo apenas sua densidade sem prever o rumo que essa solu��o est�
tomando. Para corrigir isso, � apresentado um m�todo baseado na id�ia
de densidade dos pontos, mas que utiliza de aspectos computa��o adaptativa
com uma contru��o-desconstru��o peri�dica. Essa meta-heuristica construtiva
apresenta soluc�oes muito boas em uma itera��o, sendo as melhoras
sobre a solu��o dada por uma procura no espa�o do problema atrav�s
de altera��es nas dist�ncias e e gera��o de novas solu��es.

Para a etapa construtiva da solu��o, existem os seguintes m�todos:
\begin{itemize}
\item EncontraVizinhos: Encontra os $m_{i}$vizinhos mais pr�ximos de ponto
$i$, sendo que $m_{i}\leq\frac{n}{p}$ e todos os pontos $Y_{i}$
(conjunto de pontos vizinhos do centro $i$) tem a soma de suas demandas
inferior ou igual a capacidade do agrupamento $i$.
\item CalculaDensidade: Com o conjunto de pontos vizinhos do centro $i$,
conseguimos calcular a densidade do centro atrav�s de
\end{itemize}
\[
D_{i}=\frac{m_{i}}{T(a_{i},Y_{i})}\]


sendo $T(a_{i},Y_{k})$ a fun��o que devolve a dist�ncia total de
do ponto $a_{i}$ at� todos os pontos de $Y_{k}$.
\begin{itemize}
\item CalculaArrependimento: Ap�s encontrados os pontos candidatos a centros,
� necess�rio definir os pontos que ser�o atribuidos a cada um dos
agrupamentos. Para essa inser��o, � utilizada essa fun��o para calcular
o arrependimento de associar um n� com um centro que n�o seja o mais
pr�ximo. Para um ponto $i$, tendo os pontos $j_{1}\; e\; j_{2}$
como o centro mais pr�ximo a $i$ e segundo mais pr�ximo, respectivamente,
calculamos o arrependimento de associar $i$ com um centro que n�o
seja $j_{1}$ como sendo:
\end{itemize}
\[
R_{i}=d_{ij_{2}}-\; d_{ij_{1}}\]

\begin{itemize}
\item EncontreOsMelhoresAgrupamentos (C, A): tendo A como um conjunto de
pontos e C o conjunto de centros. Essa fun��o tenta associar os pontos
em A com os agrupamentos em C e est� descrita em \ref{alg:;EncontreOsMelhoresAgrupamentos}.
\end{itemize}
%
\begin{algorithm}[h]
\caption{EncontreOsMelhoresAgrupamentos(C,A)\label{alg:;EncontreOsMelhoresAgrupamentos}\protect \\
}


$t=1$;

Minor $\leftarrow$ M�ximo de itera��es;

$alterado\leftarrow Verdadeiro$;

Enquanto ($alterado=Verdadeiro$ E $t<Minor$)

Fa�a:

~~~~~~$alterado\leftarrow False$;

~~~~~~1: Atribua todos os centros aos seus agrupamentos;

~~~~~~2: Calcule o arrependimento de todos os pontos em $A$;

~~~~~~3: Encontre o maior $R_{i}$ n�o atribuido;

~~~~~~4: Atribua $i$ para o agrupamento mais pr�ximo dispon�vel;

~~~~~~5: Atualiza $A$ com os pontos atribuidos;

~~~~~~6: Se $A\neq\emptyset$, v� para 2;

~~~~~~Para cada $C_{j}$ e seu centro $c_{j}$

~~~~~~~~~~~para cada $a_{i}\in C_{j}$

~~~~~~~~~~~~~~~~Se ($T(a_{i},C_{j})$$<T(c_{j},C_{j})$)

~~~~~~~~~~~~~~~~~~~~Atualize o novo centro de
$C_{j}$ como $a{}_{i}$;

~~~~~~~~~~~~~~~~~~~~$alterado\leftarrow Verdadeiro$;

~~~~~~~~~~~~~~~~FimSe;

~~~~~~~~~~FimPara;

~~~~~~FimPara;

~~~~~~$t\leftarrow t+1$;

FimFun��o.
\end{algorithm}


Com as fun��es definidas, um procedimento principal � chamado e apresenta
ao final o conjunto dos agrupamentos $C$. Em \ref{alg:ProcedimentoDensidade}
� apresentado tal procedimento o qual implementa a computa��o adaptativa.
De modo iterativo, esse m�todo vai construindo um agrupamento por
vez e recalculando a densidade ($D)$ e arrependimento ($R$) do conjunto.

%
\begin{algorithm}[h]
\caption{{ProcedimentoPrincipal}\label{alg:ProcedimentoDensidade}\protect \\
}


I: conjunto de $n$ pontos

C: conjunto de centros

X: conjunto de pontos n�o atribuidos em A

Z: conjunto de pontos atribuidos at� itera��o $k$

$Y_{i}$: conjunto dos $m_{i}$ pontos pr�ximos a $a_{i}$, dado pela
fun��o $EncontraVizinhos$

Passo 1:

$k=0;$

$X=A,\, C=\emptyset,\, Z=\emptyset$

Passo 2:

$k\leftarrow k+1;$

Para cada $a_{i}\in X$, encontre o conjunto $m_{i}$, $Y_{i}\subseteq X$;

Calcule a densidade ($D_{i}$) para cada $a_{i}\in X$;

Selecione o maior $D_{i}$ e atribua $a_{i}$ como centro do centro
do $k$-�simo agrupamento, $C_{k}=C_{k}\cup a_{i}$;

Elimine os pontos selecionados das pr�ximas sele��es: $X=X\setminus Y_{i}$,
$Y=Y\cup Y_{i}$;

Se $k\geq2$ ent�o chame $EncontreOsMelhoresAgrupamentos(C,Y)$;

Se $k<p$ ent�o volte ao passo 2;

$EncontreOsMelhoresAgrupamentos(C,A)$
\end{algorithm}


Uma compara��o entre as solu��es geradas pelas heur�sticas pode ser
visto na figura \ref{fig:CCP-compara=0000E7=0000E3o}. A Inst�ncia
� a mesma aplicada em ambos casos. Como podemos ver, a heur�stica
de densidade apresenta um solu��o melhor, apesar do tempo de processamento
ser muito alto pela complexidade do algoritmo.

% %
% \begin{figure}[h]
% \begin{centering}
% \includegraphics[scale=0.5]{imagens/cluster_Density_SJC4a}\includegraphics[scale=0.5]{imagens/cluster_Farthest_SJC4a}
% \par\end{centering}
% 
% \caption{\label{fig:CCP-compara=0000E7=0000E3o}A solu��o gerada pela heur�stica
% density (esquerda) possui os agrupamentos bem mais compactos que na
% solu��o pela heur�stica farthest (direita).}
% 
% 
% 
% \end{figure}



\subsection{CCCP}

� proposto como heur�sticas de solu��o para esse problema 2 m�todos
contrutivos para gerar uma primeira solu��o e a aplica��o de uma VNS
(variable neightbourhood search) para melhoria dessa solu��o. Ser�
apresentado as duas heur�sticas da fase contrutiva, sendo a segunda
fase de melhorias apenas altera��es na solu��o efetuando troca ou
movimento aleat�rio de pontos de um agrupamento para outro por um
periodo de tempo.


\subsubsection{q-Trees Balanceadas}

Dado o conjunto de pontos no um espa�o $\Re^{2}$, eles s�o inseridos
em um estrutura q-tree (Quadrant-Tree), a qual se ajusta, atrav�s
de rota��es semelhantes as �rvores AVL, de modo que as ra�zes das
sub �rvores s�o a mediana dos pontos pertencentes a elas. Essa primeira
parte define as parti��es para servirem como entrada para os m�todos
Forgy/H-Means+, referenciado por Forgy no decorrer do relat�rio, e
JMeans.

Para defini��o das parti��es apartir da q-tree, s�o usados duas estrat�gias:
Next Fit, onde o n� ra�z da arvore � �nserido no agrupamento atual
ou, se n�o isso ultrapassar a capacidade do agrupamento, � inserido
em um novo agrupamento, e Best Fit que insere o n� raiz da q-tree
no cluster que possui o centroide mais pr�ximo ou se n�o � poss�vel
inserir em nenhum agrupamento existente, � criado um novo. Toda vez
que o n� raiz � inserido em algum agrupamento, ele � retirado da q-tree
a qual se ajusta passando outro n� para raiz.

Os m�todos Forgy e JMeans s�o apresentados em \ref{alg:Forgy} e \ref{alg:JMeans}
respectivamente.

%
\begin{algorithm}[h]
\caption{Forgy\label{alg:Forgy}}


$k\leftarrow0;$

Passo 1:

Se $k=0$ ent�o construa $C_{1}^{0},\ldots,C_{p}^{0}$

sen�o construa novos $C_{1}^{k},\ldots,C_{p}^{k}$, atribuindo os
indiv�duos ao centro dispon�vel mais pr�ximo.

$k\leftarrow k+1$;

Calcule os centroides de $C_{1}^{k},\ldots,C_{p}^{k}$;

Se $NAC\leq p$ ent�o

~~~~Ordene os pontos pela ordem decrescente dos centroides de
seus agrupamentos;

~~~~Selecione os primeiros $p-NAC$ pontos e transforme em novos
centroides;

~~~~Reduza de $f(C_{m}^{k})$ as distancias selecionadas;

FimSe

Se $f(C_{m}^{k})=f(C_{m}^{k})$ ent�o pare;

sen�o, volte ao Passo 1;
\end{algorithm}


%
\begin{algorithm}[h]
\caption{JMeans\label{alg:JMeans}}


$k\leftarrow0;$

Se $k=0$ ent�o construa $C_{1}^{0},\ldots,C_{p}^{0}$;

Passo 1:

Atribua pontos de $C_{i}$ para um outro centro $C_{j}$ ($j\neq i$).
Esses pontos devem ser afastados dos centroides por uma toler�ncia
$\varepsilon$, marque eles como n�o ocupados.

Para todo $j\in I$ fa�a

~~~~~~Adicione um novo agrupamento com centroide $\bar{g}_{M+1}$em
alguma entidade n�o ocupada $a_{j}$ e o �ndice $i$ do melhor centro
removido; $v_{il}$ indicam a altera��o na fun��o objetivo;

~~~~~~Mantenha o par de �ndices $i^{\prime}$e $j^{\prime}$,
onde $v_{ij}$� m�nimo;

~~~~~~Troque o centroide $g_{i^{\prime}}$por $g_{j\prime}$e
atualize as atribui��es de modo a ter uma nova parti��o $P_{M}^{\prime}$;

~~~~~~$f^{\prime}\leftarrow f_{opt}+v_{i^{\prime}j^{\prime}}$;

FimPara

Se N�o houve melhora, termine.

Caso contr�rio, implemente a maior parti��o, retorne ao passo 1;
\end{algorithm}


O valor $v_{ij}$ utilizado no m�todo JMeans representa a melhora
na fun��o objetivo pela realoca��o do indiv�duo $i$ e � dado pela
f�rmula em \eqref{eq:JMeansVij}, tendo que $\bar{g}_{i}$ � o centroide
do agrupamento $C_{i}$, $a_{j}\in I$ e $n_{i}$ � o valor da cardinalidade
do agrupamento $C_{i}$. O Valor de $NAC$ no m�todo Forgy, � o n�mero
atual de agrupamentos em uso.

\begin{equation}
v_{ij}\leftarrow\frac{n_{i}}{n_{i}+1}\Vert\overline{g_{i}}-a_{j}\Vert^{2}-\frac{n_{l}}{n_{l}+1}\Vert\overline{g_{l}}-a_{j}\Vert^{2},\;\; a_{j}\in C_{l},\;\;\overline{g_{i}}\notin C_{l}\label{eq:JMeansVij}\end{equation}



\subsubsection{Unconstrained to Constrained}

A segunda heur�stica apresentada resolve o problema n�o levando em
considera��o a capacidade de cada agrupamento para gerar uma solu��o
inicial a qual pode ser uma solu��o para o CCCP infact�vel.

A partir da solu��o inicial, o pr�ximo passo � verificar se ela �
fact�vel e caso n�o seja, localizar o indiv�duo que sobrecarrega algum
agrupamento, e move-lo para outro que suporte ele. Caso n�o exista
nenhum agrupamento que suporte a sua inser��o, ent�o � aberto um novo
agrupamento. Para gerar uma solu��o inicial, � utilizado a mesma fun��o
forgy apresentada acima, mas n�o levando em considera��o capacidade
dos agrupamentos. Ap�s a execu��o dessa fun��o, todos os agrupamentos
s�o percorridos e para os que tem maior demanda que a sua capacidade,
tem um indiv�duo que o sobrecarrega movido para outro agrupamento
que o suporte e que seja o mais pr�ximo de si.


\section{PDOS visto como um problema de agrupamento}

Para atender um conjunto de solicita��es dos clientes de forma mais
r�pida, as equipes devem diminuir ao m�ximo o tempo de deslocamento
entre cada atendimento, uma vez que o tempo do atendimento n�o pode
ser reduzido sem comprometer a qualidade do servi�o prestado. Diminuir
o tempo de deslocamento entre os atendimentos pode ser obtido pela
designa��o das tarefas para as equipes de modo que as tarefas designadas
a uma dada equipe fiquem pr�ximas entre si, diminuindo assim a dist�ncia
entre elas.

Com tais caracter�sticas, o PDOS pode ser modelado como um problema
de agrupamento capacitado, onde a capacidade prov�m da disponibilidade
de tempo que uma equipe disp�em por dia, normalmente 8 horas, e as
demandas o tempo necess�rio para resolver os servi�os. Para representa��o
mais fiel a realidade, deve-se considerar as rotas e tamb�m o tempo
necess�rio para se deslocar de um atendimento at� outro. As rotas
s�o calculadas dentre os atendimentos de cada equipe como um tour,
passando por todos os atendimentos, n�o sendo necess�rio retornar
ao ponto de origem da rota e o tempo de deslocamento � calculado normalmente
pelo c�lculo $d_{ij}*K$, onde $d_{ij}$ � a dist�ncia do atendimento
$i$ at� o atendimento $j$ e $K$ � uma constante de quanto tempo(minutos
por exemplo) � necess�rio para percorrer uma unidade de medida (Km
por exemplo).

A representa��o de tempo de deslocamento e o c�lculo das rotas, n�o
foram abordadas nessa primeira parte do trabalho, bem como tamb�m
o tratamento das prioridades de atendimento ficaram para a segunda
parte do projeto.

Para a solu��o de problemas PDOS, dado os dois modelos de problema
de agrupamento estudados, vimos que seria mais representativo o modelo
$CCP$, mas a heur�stica apresentada em \cite{negreiros2006capacitated},
que faz uso das q-trees, pode ser utilizada como forma de resolver
problemas do tipo $CCP$ sendo isso citado no pr�prio artigo que foram
testados em inst�ncias de problemas $CCP$, mas n�o � indicado os
valores de tal heur�stica aplicada ao $CCP$.
\chapter{Algoritmos Implementados}

Foram implementados os m�todos construtivos propostos por \cite{ahmadi2004density,osman1994capacitated}
mas n�o as etapas de refinamento propostas por ele. Foi escolhido
o paradigma de programa��o orientado a objetos pela facilidade de
manuten��o e modularidade dos c�digos al�m da facilidade escrever
casos de testes de modo a perceber regress�es de forma mais f�cil.
O diagrama de classes UML, bem como a associa��o entre bibliotecas
do sistema, se encontram na sec��o de anexos ao final do relat�rio.

Foi utilizada a linguagem de programa��o C++ com o compilador do GNU
(g++), o CMake como gerador de arquivos Makefile (arquivos que cont�m
informa��es de como compilar e testar o programa), a biblioteca Qt
para parte de testes e algumas fun��es dispon�veis nela tais como
leitura de arquivos, estruturas de dados e parte gr�fica para mostrar
de forma bem simples o resultado final.

Como gerenciamento de vers�o do projeto foi utilizado o Git, que �
um SCV distribu�do e assim permite o seu uso mesmo quando n�o se possu�
acesso ao servidor. Para servidor, foi utilizado o servidor github,
e os c�digos podem ser acessados pelo endere�o www.github.com/wiglot/CCP.
\chapter{Pr�ximas etapas}

Para a segunda parte deste trabalho, ser� dada continuidade nos estudos
de formas de resolver o problema, procurando por artigos que tenham
alguma solu��o pr�xima desse trabalho para darmos continuidade no
desenvolvimento de heur�sticas para solu��o do problema.

Para essa etapa tamb�m temos em vista o uso de caracter�sticas que
ainda n�o foram abordadas tais como prioridades de atendimento, onde
ser� desejado que os atendimentos com maior prioridade fiquem em agrupamentos
distintos, de modo a serem atendidos por diferentes equipes, evitando
que uma equipe assuma todos atendimentos de urg�ncia em uma localidade
por exemplo. Como o deslocamento tamb�m deve ser levado em considera��o,
iremos calcular as rotas dentro de cada agrupamento para as equipes
e de posse dessa rota, iremos calcular o tempo necess�rio para a equipe
efetuar todo deslocamento, o qual ser� adicionado aos tempos de atendimentos
associados com essa equipe. Esse c�lculo de tempo de deslocamento,
traz um problema que � a necessidade de se calcular a rota para cada
ponto inserido em um agrupamento, leva a necessidade de reestruturar
a rota de tal equipe. 
% Conclus�o - Obrg
\chapter*{Conclus�es}

Al�m de apresentar um breve
O que contribuimos com esse trabalho? %Inclui arquivo conclusao.tex

% ELEMENTOS P�S-TEXTUAIS
% Refer�ncias - Obrg
\bibliography{../bibitex/bibliography_CCP} %Seu arquivo Bibitex

% Gloss�rio
% Ap�ndice(s)

% Anexo(s)
\appendix

\part*{Anexos}


\chapter*{Diagramas UML}

%
% \begin{figure}[h]
% \includegraphics[scale=0.5]{imagens/Instance}
% 
% \caption{Classes que representam uma Inst�ncia e as solu��es }
% 
% \end{figure}


%
% \begin{figure}[h]
% \includegraphics[scale=0.6]{imagens/Algorthms}
% 
% \caption{Classes dos algoritmos das heur�sticas de constru��o}
% 
% 
% 
% \end{figure}


\begin{landscape}


\chapter*{Associa��o dos M�dulos}

%
\begin{figure}[h]

\tikzset{
  arrow/.style={-stealth', line width=0.5pt},
  every picture/.append style={line width=1pt},
}

\tiny {

\enlargethispage{100cm}
% Start of code
% \begin{tikzpicture}[anchor=mid,>=latex',join=bevel,]
\begin{tikzpicture}[>=latex',join=bevel,scale=0.65]
  \pgfsetlinewidth{1bp}
%%
\pgfsetcolor{black}
  % Edge: node0 -> node4
  \draw [->] (130.06bp,219.43bp) .. controls (125.48bp,210.75bp) and (119.51bp,199.45bp)  .. (109.32bp,180.13bp);
  % Edge: node9 -> node12
  \draw [->] (716.02bp,147.39bp) .. controls (712.23bp,146.19bp) and (708.43bp,145.04bp)  .. (704.75bp,144bp) .. controls (632.92bp,123.76bp) and (612.69bp,127.83bp)  .. (540.75bp,108bp) .. controls (537.36bp,107.07bp) and (533.87bp,106.04bp)  .. (520.57bp,101.88bp);
  % Edge: node0 -> node11
  \draw [->] (108.83bp,226.23bp) .. controls (86.415bp,218.52bp) and (56.87bp,204.27bp)  .. (42.75bp,180bp) .. controls (31.694bp,161bp) and (36.774bp,136.1bp)  .. (47.572bp,107.83bp);
  % Edge: node6 -> node1
  \draw [->] (392.12bp,147.43bp) .. controls (390.6bp,139.01bp) and (388.64bp,128.14bp)  .. (385.02bp,108.13bp);
  % Edge: node9 -> node3
  \draw [->] (730.64bp,147.43bp) .. controls (710.16bp,136bp) and (681.55bp,120.03bp)  .. (650.79bp,102.86bp);
  % Edge: node8 -> node2
  \draw [->] (661.03bp,147.43bp) .. controls (677.35bp,137.64bp) and (699.21bp,124.52bp)  .. (726.54bp,108.13bp);
  % Edge: node5 -> node1
  \draw [->] (288.43bp,147.43bp) .. controls (304.26bp,137.68bp) and (325.46bp,124.64bp)  .. (352.29bp,108.13bp);
  % Edge: node2 -> node10
  \draw [->] (723.61bp,74.684bp) .. controls (720.65bp,73.658bp) and (717.67bp,72.741bp)  .. (714.75bp,72bp) .. controls (615.37bp,46.802bp) and (311.96bp,27.968bp)  .. (174.49bp,20.41bp);
  % Edge: node0 -> node1
  \draw [->] (143.2bp,219.19bp) .. controls (151.34bp,199.15bp) and (168.56bp,163.65bp)  .. (194.75bp,144bp) .. controls (234.16bp,114.43bp) and (289.34bp,101.03bp)  .. (339.68bp,93.545bp);
  % Edge: node5 -> node12
  \draw [->] (310.78bp,147.35bp) .. controls (344.57bp,136.56bp) and (391.55bp,121.49bp)  .. (432.75bp,108bp) .. controls (435.9bp,106.97bp) and (439.14bp,105.9bp)  .. (452.2bp,101.58bp);
  % Edge: node9 -> node2
  \draw [->] (756.75bp,147.43bp) .. controls (756.75bp,139.1bp) and (756.75bp,128.37bp)  .. (756.75bp,108.13bp);
  % Edge: node4 -> node1
  \draw [->] (147.76bp,147.85bp) .. controls (152.48bp,146.52bp) and (157.2bp,145.22bp)  .. (161.75bp,144bp) .. controls (218.97bp,128.68bp) and (285.01bp,112.7bp)  .. (339.44bp,99.83bp);
  % Edge: node8 -> node3
  \draw [->] (634.93bp,147.43bp) .. controls (633.86bp,138.89bp) and (632.48bp,127.82bp)  .. (629.95bp,107.6bp);
  % Edge: node6 -> node12
  \draw [->] (413.37bp,147.43bp) .. controls (426.18bp,137.4bp) and (443.46bp,123.88bp)  .. (466.24bp,106.05bp);
  % Edge: node7 -> node3
  \draw [->] (535.22bp,147.43bp) .. controls (553.25bp,136.24bp) and (578.27bp,120.71bp)  .. (606.27bp,103.33bp);
  % Edge: node6 -> node3
  \draw [->] (441.9bp,147.43bp) .. controls (483.67bp,134.52bp) and (544.13bp,115.84bp)  .. (594.23bp,100.36bp);
  % Edge: node4 -> node10
  \draw [->] (103.3bp,143.76bp) .. controls (108.09bp,119.09bp) and (116.69bp,74.86bp)  .. (124.23bp,36.09bp);
  % Edge: node0 -> node2
  \draw [->] (166.67bp,232.82bp) .. controls (294.44bp,227.45bp) and (803bp,204.6bp)  .. (825.75bp,180bp) .. controls (845.2bp,158.96bp) and (820.23bp,132.75bp)  .. (786.59bp,108.02bp);
  % Edge: node5 -> node3
  \draw [->] (320.06bp,147.35bp) .. controls (325.01bp,146.17bp) and (329.97bp,145.03bp)  .. (334.75bp,144bp) .. controls (425.61bp,124.41bp) and (449.45bp,125.42bp)  .. (540.75bp,108bp) .. controls (552.59bp,105.74bp) and (565.3bp,103.19bp)  .. (587.08bp,98.686bp);
  % Edge: node9 -> node1
  \draw [->] (717.25bp,147.4bp) .. controls (713.06bp,146.14bp) and (708.84bp,144.97bp)  .. (704.75bp,144bp) .. controls (589.21bp,116.65bp) and (553.22bp,135.4bp)  .. (423.85bp,105.62bp);
  % Edge: node7 -> node2
  \draw [->] (557.1bp,147.39bp) .. controls (561.03bp,146.21bp) and (564.95bp,145.07bp)  .. (568.75bp,144bp) .. controls (630.08bp,126.78bp) and (649.56bp,128.97bp)  .. (723.47bp,104.84bp);
  % Edge: node7 -> node12
  \draw [->] (506.69bp,147.43bp) .. controls (503.72bp,138.86bp) and (499.86bp,127.75bp)  .. (492.95bp,107.86bp);
  % Edge: node4 -> node11
  \draw [->] (88.899bp,143.83bp) .. controls (83.973bp,135.58bp) and (78.049bp,125.66bp)  .. (67.448bp,107.91bp);
  % Edge: node0 -> node3
  \draw [->] (166.8bp,232.75bp) .. controls (292.79bp,227.2bp) and (786.63bp,203.97bp)  .. (808.75bp,180bp) .. controls (819.6bp,168.24bp) and (818.31bp,156.83bp)  .. (808.75bp,144bp) .. controls (800.64bp,133.12bp) and (727.4bp,113.75bp)  .. (666.77bp,99.032bp);
  % Edge: node8 -> node1
  \draw [->] (581.63bp,147.41bp) .. controls (530.71bp,133.94bp) and (459.89bp,115.19bp)  .. (423.96bp,105.19bp);
  % Edge: node8 -> node12
  \draw [->] (606.39bp,147.43bp) .. controls (582.94bp,136.17bp) and (550.32bp,120.52bp)  .. (515.79bp,103.94bp);
  % Edge: node0 -> node10
  \draw [->] (108.77bp,225.48bp) .. controls (85.409bp,217.26bp) and (53.194bp,202.69bp)  .. (32.75bp,180bp) .. controls (9.7676bp,154.49bp) and (11.08bp,141.75bp)  .. (4.7496bp,108bp) .. controls (1.7995bp,92.274bp) and (-3.9994bp,85.396bp)  .. (4.7496bp,72bp) .. controls (20.241bp,48.28bp) and (48.728bp,34.912bp)  .. (84.051bp,24.816bp);
  % Edge: node7 -> node1
  \draw [->] (485.44bp,147.43bp) .. controls (467.6bp,137.55bp) and (443.66bp,124.29bp)  .. (414.48bp,108.13bp);
  % Node: node11
\begin{scope}
  \definecolor{strokecol}{rgb}{0.0,0.0,0.0};
  \pgfsetstrokecolor{strokecol}
  \draw (57bp,90bp) ellipse (43bp and 18bp);
  \draw (56.75bp,90bp) node {libQtGui.so};
\end{scope}
  % Node: node10
\begin{scope}
  \definecolor{strokecol}{rgb}{0.0,0.0,0.0};
  \pgfsetstrokecolor{strokecol}
  \draw (128bp,18bp) ellipse (47bp and 18bp);
  \draw (127.75bp,18bp) node {libQtCore.so};
\end{scope}
  % Node: node12
\begin{scope}
  \definecolor{strokecol}{rgb}{0.0,0.0,0.0};
  \pgfsetstrokecolor{strokecol}
  \draw (487bp,90bp) ellipse (45bp and 18bp);
  \draw (486.75bp,90bp) node {libQtTest.so};
\end{scope}
  % Node: node9
\begin{scope}
  \definecolor{strokecol}{rgb}{0.0,0.0,0.0};
  \pgfsetstrokecolor{strokecol}
  \draw (800bp,168bp) -- (757bp,180bp) -- (714bp,168bp) -- (714bp,147bp) -- (800bp,147bp) -- cycle;
  \draw (756.75bp,162bp) node {CCPTest};
\end{scope}
  % Node: node8
\begin{scope}
  \definecolor{strokecol}{rgb}{0.0,0.0,0.0};
  \pgfsetstrokecolor{strokecol}
  \draw (696bp,168bp) -- (637bp,180bp) -- (578bp,168bp) -- (578bp,147bp) -- (696bp,147bp) -- cycle;
  \draw (636.75bp,162bp) node {CCPSolution};
\end{scope}
  % Node: node1
\begin{scope}
  \definecolor{strokecol}{rgb}{0.0,0.0,0.0};
  \pgfsetstrokecolor{strokecol}
  \draw (424bp,108bp) -- (340bp,108bp) -- (340bp,72bp) -- (424bp,72bp) -- cycle;
  \draw (381.75bp,90bp) node {CCPModelLib};
\end{scope}
  % Node: node0
\begin{scope}
  \definecolor{strokecol}{rgb}{0.0,0.0,0.0};
  \pgfsetstrokecolor{strokecol}
  \draw (167bp,240bp) -- (138bp,252bp) -- (109bp,240bp) -- (109bp,219bp) -- (167bp,219bp) -- cycle;
  \draw (137.75bp,234bp) node {CCP};
\end{scope}
  % Node: node3
\begin{scope}
  \definecolor{strokecol}{rgb}{0.0,0.0,0.0};
  \pgfsetstrokecolor{strokecol}
  \draw (628bp,108bp) -- (550bp,90bp) -- (628bp,72bp) -- (706bp,90bp) -- cycle;
  \draw (627.75bp,90bp) node {CCPAlgorithms};
\end{scope}
  % Node: node2
\begin{scope}
  \definecolor{strokecol}{rgb}{0.0,0.0,0.0};
  \pgfsetstrokecolor{strokecol}
  \draw (790bp,108bp) -- (724bp,108bp) -- (724bp,72bp) -- (790bp,72bp) -- cycle;
  \draw (756.75bp,90bp) node {CCPIOLib};
\end{scope}
  % Node: node5
\begin{scope}
  \definecolor{strokecol}{rgb}{0.0,0.0,0.0};
  \pgfsetstrokecolor{strokecol}
  \draw (326bp,168bp) -- (265bp,180bp) -- (204bp,168bp) -- (204bp,147bp) -- (326bp,147bp) -- cycle;
  \draw (264.75bp,162bp) node {CCPDistance};
\end{scope}
  % Node: node4
\begin{scope}
  \definecolor{strokecol}{rgb}{0.0,0.0,0.0};
  \pgfsetstrokecolor{strokecol}
  \draw (148bp,180bp) -- (52bp,180bp) -- (52bp,144bp) -- (148bp,144bp) -- cycle;
  \draw (99.75bp,162bp) node {CCPClusterView};
\end{scope}
  % Node: node7
\begin{scope}
  \definecolor{strokecol}{rgb}{0.0,0.0,0.0};
  \pgfsetstrokecolor{strokecol}
  \draw (560bp,168bp) -- (512bp,180bp) -- (464bp,168bp) -- (464bp,147bp) -- (560bp,147bp) -- cycle;
  \draw (511.75bp,162bp) node {CCPRead};
\end{scope}
  % Node: node6
\begin{scope}
  \definecolor{strokecol}{rgb}{0.0,0.0,0.0};
  \pgfsetstrokecolor{strokecol}
  \draw (446bp,168bp) -- (395bp,180bp) -- (344bp,168bp) -- (344bp,147bp) -- (446bp,147bp) -- cycle;
  \draw (394.75bp,162bp) node {CCPModel};
\end{scope}
%
\end{tikzpicture}
% End of code

}

\caption{Bibliotecas que fazem parte do sistema. Arestas apontam depend�ncias
diretas de um m�dulo por outro.}

\end{figure}


\end{landscape}



\end{document}



