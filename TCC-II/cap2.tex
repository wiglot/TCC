\chapter{Algoritmos Implementados}

Foram implementados os m�todos construtivos propostos por \cite{ahmadi2004density,osman1994capacitated}
mas n�o as etapas de refinamento propostas por ele. Foi escolhido
o paradigma de programa��o orientado a objetos pela facilidade de
manuten��o e modularidade dos c�digos al�m da facilidade escrever
casos de testes de modo a perceber regress�es de forma mais f�cil.
O diagrama de classes UML, bem como a associa��o entre bibliotecas
do sistema, se encontram na sec��o de anexos ao final do relat�rio.

Foi utilizada a linguagem de programa��o C++ com o compilador do GNU
(g++), o CMake como gerador de arquivos Makefile (arquivos que cont�m
informa��es de como compilar e testar o programa), a biblioteca Qt
para parte de testes e algumas fun��es dispon�veis nela tais como
leitura de arquivos, estruturas de dados e parte gr�fica para mostrar
de forma bem simples o resultado final.

Como gerenciamento de vers�o do projeto foi utilizado o Git, que �
um SCV distribu�do e assim permite o seu uso mesmo quando n�o se possu�
acesso ao servidor. Para servidor, foi utilizado o servidor github,
e os c�digos podem ser acessados pelo endere�o www.github.com/wiglot/CCP.