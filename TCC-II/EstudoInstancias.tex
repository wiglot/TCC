\chapter{Estudo das Caracter�sticas das Inst�ncias}
chamamos de folga a diferen�a entre a soma das demanda e a capacidade total oferecida. Para comparar entre as diferentes inst�ncias, utilizamos a formula \ref{folga}:
% \begin{equation}
\begin{align}
 F &= \frac{\sum_{p\in P} p_{cap} }{\sum_{i \in I} i_{dem}} \label{folga}
\end{align}

Onde $i_{dem}$ � a demanda do indiv�duo $i$ e $p_{cap}$ a capacidade do agrupamento $p$. $F$ � maior que 1 se a demanda total � menor que a capacidade total. Para valores de $F$ muito pr�ximos a 1, mas maiores que 1, as heur�sticas tendem a n�o conseguir nem mesmo uma solu��o fact�vel, principalmente heur�sticas que utilizam de procedimentos gulosos para aloca��o de indiv�duos, como por exemplo a escolha do agrupamento com mediana mais pr�xima do indiv�duo a ser alocado. Para valores menores que 1 n�o existe uma solu��o v�lida para a inst�ncia, uma vez que existe maior demanda que oferta (capacidade).

Para o caso de $ \sum_{i \in I} i_{dem} \geq p_{cap},\forall p \in P $ podemos dizer que o problema � equivalente ao problemas das P-medianas n�o capacitado, uma vez que a capacidade n�o chega a ser uma restri��o no problema e todos os indiv�duos podem ser associados a apenas uma agrupamento.

Essa ultima coloca��o mostra o quanto o valor de folga pode interferir na gera��o de solu��es das inst�ncias.
Nas inst�ncias in�ditas apresentadas aqui, modificamos a
%  F =
% \end{equation}


Mostrar do estudo da folga, diferen�a de demanda entre os pontos (min, m�x)

Apresentar tamb�m estudo de dispers�o dos pontos usando centro de massa com vari�ncia da capacidade dos pontos criados, e dist�ncia entre os pontos e o centro de massa (ser� que vai dar tempo??).

