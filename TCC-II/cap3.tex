\chapter{Pr�ximas etapas}

Para a segunda parte deste trabalho, ser� dada continuidade nos estudos
de formas de resolver o problema, procurando por artigos que tenham
alguma solu��o pr�xima desse trabalho para darmos continuidade no
desenvolvimento de heur�sticas para solu��o do problema.

Para essa etapa tamb�m temos em vista o uso de caracter�sticas que
ainda n�o foram abordadas tais como prioridades de atendimento, onde
ser� desejado que os atendimentos com maior prioridade fiquem em agrupamentos
distintos, de modo a serem atendidos por diferentes equipes, evitando
que uma equipe assuma todos atendimentos de urg�ncia em uma localidade
por exemplo. Como o deslocamento tamb�m deve ser levado em considera��o,
iremos calcular as rotas dentro de cada agrupamento para as equipes
e de posse dessa rota, iremos calcular o tempo necess�rio para a equipe
efetuar todo deslocamento, o qual ser� adicionado aos tempos de atendimentos
associados com essa equipe. Esse c�lculo de tempo de deslocamento,
traz um problema que � a necessidade de se calcular a rota para cada
ponto inserido em um agrupamento, leva a necessidade de reestruturar
a rota de tal equipe.