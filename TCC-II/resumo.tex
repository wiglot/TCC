\begin{resumo}
$\phantom{linha em branco}$\\
\noindent Com uma concorr�ncia cada vez maior, as empresas devem tornar suas opera��es cada vez mais enxutas para evitar gastos desnecess�rios. Para empresas que atendem clientes distribu�dos por uma cidade, esses atendimentos devem ser feitos de maneira otimizada, evitando desperd�cios, tanto financeiros quanto de tempo, com deslocamentos desnecess�rios. Para evitar tais deslocamentos, o despacho dos atendimentos para as equipes  deve levar em considera��o a localiza��o dos atendimentos e o tempo necess�rio para executar tal atendimento. Para representar isso, podemos olhar esse problema como um problema de agrupamento capacitado, onde desejamos associar os atendimentos com as equipes de modo que todos atendimentos associados a uma equipe estejam pr�ximos uns dos outros e que seja poss�vel de executar todos eles dentro de sua jornada de trabalho.

\noindent Este trabalho faz um estudo de alguns m�todos heur�sticos cl�ssicos na literatura para o problema de agrupamento capacitado e de busca local aplicados ao problema de despacho de ordens de servi�o (PDOS), tratando tamb�m sobre algumas caracter�sticas das inst�ncias como a capacidade total excedente e a dispers�o dos atendimentos no espa�o. Al�m disso, n�s desenvolvemos um pequena altera��o em um dos m�todos j� existentes, de modo que o m�todo resultante conseguiu produzir solu��es melhores que que obtidas com o m�todo original para a grande maioria das inst�ncias.\\
$\phantom{linha em branco}$\\
\begin{espacosimples}
Palavras-chave: PO; Heur�sticas; PAC, Agrupamento.
\end{espacosimples}
\end{resumo}

% *********** ABSTRACT ************
\begin{abstract}
$\phantom{linha em branco}$\\
\noindent With competition increasing, companies must make their operations more lean to avoid unnecessary expenses. For companies that serve customers spread over a city, these services must be made optimally, avoiding waste, both financial and time, with unnecessary travels. To avoid such displacement, dispatching of calls for the teams must take into account the location of call in space and time required to perform such call. To represent this, we can look at this problem as a capacitated clustering problem, where we want to associate the calls with the teams so that all calls associated with a team are close to each other, reducing the distance between the calls, and that it is possible they all run within team's workday.

\noindent This work is a study of some classical heuristics methods in the literature for the clustering problem and local search apllyed to the service order dispatch problem (SODP), dealing also with some characteristics of the instances as the total capacity surplus and dispersion of care in space. In addition, we developed a small change in one of the existing methods, so that the resulting method could produce better solutions than that obtained with the original method to the vast majority of instances.\\
$\phantom{linha em branco}$\\
\begin{espacosimples}
Key-words: OR; Heuristic; CCP, Clustering;
\end{espacosimples}
\end{abstract}