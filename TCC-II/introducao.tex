\chapter*{Introdu��o}
Em alguns setores de empresas, para atender um cliente, elas devem deslocar um funcion�rio at� a casa do cliente para executar o servi�o.
Isso ocorre principalmente em empresas que prestam algum tipo de servi�o de atendimento domiciliar aos clientes, como por exemplo empresas de entrega de correspond�ncia, provedores de internet no que diz suporte aos clientes e companias de distribui��o de energia el�trica.

Esses tipos de atendimentos aos clientes s�o chamados de ordens de servi�o (OS), podendo ter outros nomes conforme a �rea de atua��o da empresa, mas sempre com o intuito de registrar que a empresa deve fazer um atendimento a um determinado cliente.
Em nosso cen�rio tomaremos algumas premissas como v�lidas, como por exemplo:
Uma OS deve ser atendida por um funcion�rio (ou uma equipe de funcion�rios).
N�o levamos em considera��o agendamentos de hor�rios, apenas que todas as OS devem ser executas.
Uma equipe consegue executar todas as OS atribuidas a ela no tempo estimado e predefinido de cada uma delas.
O tempo de atendimento de todas as ordens atribuidas a uma equipe n�o deve ultrapassarem sua jornada de trabalho que � pr�viamente estipulada.

-*-Esse ultimo cen�rio que esse trabalho toma emprestado para o estudo.

Designar um conjunto dessas OS para execu��o por um funcion�rio , de modo que ele n�o precise se deslocar tanto entre cada atendimento, � algo n�o trivial quando nos deparamos um cen�rio com dezenas de equipes e centenas de OS.

-Contextualizar: Pq esse problema? O que a empresas perdem com ele? Que beneficios a solu��o dele tr�z?

-PDOS - O que � esse problema? COmo ele � representado (n�o por grafo, v�rtices, etc, mas o que � uma ordem de servi�o, o que uma equipe faz, etc)
-PDOS como CCP - Porque associamos o mesmo?

