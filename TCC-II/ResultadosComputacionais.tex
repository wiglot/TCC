\chapter{Resultados Computacionais}
Apresenta��o e an�lise dos resultados obtidos

Na tabela \ref{tab:valoresInstancias} apresentamos os valores obtidos para os 5 algoritmos implementados. Para os algoritmos que utilizam de aleatoriedade na sele��o dos pontos, eles foram executados 10 vezes cada um e foi tirada a m�dia das 10 execu��es. Mais a frente mostraremos a vari�ncia dos resultados obtidos.
\begin{table}[htp]
\begin{center}
\caption {\\Valores da obtidos das inst�ncias pelos algoritmos construtivos.\label{tab:valoresInstancias}}
\begin{tabular}{lrrrrrrr}
\toprule
Inst�ncia &\multicolumn{1}{c}{n}&\multicolumn{1}{c}{p}& \multicolumn{1}{c}{Density}
          & \multicolumn{1}{c}{Farthest} & \multicolumn{1}{c}{H-Means}
          & \multicolumn{1}{c}{J-Means}  & \multicolumn{1}{c}{Random Density} \\
\midrule
%%%%%%%Valores
SJC1  &  100&  10&18.193,2 & 25.905,0 &19.169,4 &20.247,9 &17.693,9 \\
SJC2  &  200&  15& 34.520,4  &52.121,3 &35.179,5 &34.591,2&34.147,2 \\
SJC3a &  300&  25& 47.976,7  &76.345,3 &49.231,8 &48.140,4  &45.907,9 \\
SJC3b &  300&  30& 42.501,7  &69.956,2 & 42.607,4 &42.329,0  &42.151,3 \\
SJC4a &  402&  30& 65.655,4 &103.171,0 &66.763,9  &67.595,4 &63.602,9 \\
SJC4b &  402&  40& 54.077,1  &92.948,9 & 57.092,1 &55.412,2 &54.438,2 \\
SJC5  &  402&  20& 97.681,6  &134.798,0 & 99.502,9  &100.117,0  &86.325,5 \\
AESM\_10 &  1142& 21& 96.738,2 & 20.519,8     &  15.482,0 & 16.409,9 &  15.251,7 \\
AESM\_20 &  & & 97.798,8 & 16.466,2  &  16.472,7 & 13.753,0   &  11.665,8 \\
AESM\_30 &  & & 69.541,8 & 16.582,9   &  13.504,5 &  12.859,9&  11.626,2\\
AECP\_10 &  324&  10& 27.877,4 &   6.447,6 &  6.482,0& 6.846,3   &   6.458,0 \\
AECP\_20 &  & & 9.889,6  & 6.173,6   &   5.900,1 & 6.331,2   &   5.835,2  \\
AECP\_30 &  & & 8.374,6  &6.155,2    &5.646,0 & 5.608,7 &      5.799,3 \\
AENH\_10 &  2327& 17& 86.174,4 & 24.735,9 & 26.241,5& 26.056,1 &  23.910,3 \\
AENH\_20 &  && 73.593,2   & 24.549,9 &  24.860,3   & 24.218,4  &   22.826,4 \\
AENH\_30 &  && 74.153,2&25.785,4 & 24.913,3 & 27.481,7 & 23.977,9 \\
AELJ\_10 &  646& 14& 32.398,0 & 9.174,7   &9.982,1 & 9.703,8&10.177,6 \\
AELJ\_20 &  & & 32.248,7 & 8.798,5    &8.109,5 & 8.283,7& 8.472,1 \\
AELJ\_30 &  & & 27.915,8 &8.735,7   &7.535,1  &7.645,9 & 8.086,4 \\
AEAL\_10 &  448&15& 29.700,4 & 10.850,5  &10.933,5& 11.216,0 &10.838,4 \\
AEAL\_20 &  && 27.219,8 &10.763,5   &9.164,8 & 9.336,6 & 9.026,3 \\
AEAL\_30 &  && 26.814,7 & 10.752,1  &8.837,1 & 9.026,4 & 8.818,8\\
\bottomrule
\end{tabular}
\end{center}
\end{table}

As solu��es aleat�rias tem os desvios padr�o apresentados na tabela \ref{tab:stdDev}.

\begin{threeparttable}[htp]
\begin{center}
\caption {\\Desvio padr�o dos valores obtidos pelos algoritmos com aleatoriedade.\label{tab:stdDev}}
\begin{tabular}{lrrrrrr}
\toprule
Inst�ncia & \multicolumn{2}{c}{H-Means}
          & \multicolumn{2}{c}{J-Means}  & \multicolumn{2}{c}{Random Density} \\

 & \multicolumn{1}{c}{M�dia}& \multicolumn{1}{c}{DP\tnote{1}}
 & \multicolumn{1}{c}{M�dia}& \multicolumn{1}{c}{DP}
 & \multicolumn{1}{c}{M�dia}& \multicolumn{1}{c}{DP} \\
\midrule
%%%%%%%Valores
SJC1     &20.066,6 &619,3  &21.214,0   &966,1  &18.430,9   &653,4  \\
SJC2     &36.767,7  &1.205,0    &36.109,6   &1.269,5    &34.700,3   &355,3  \\
SJC3a    &50.651,7  &1.019,7    &50.041,5   &1.631,5    &47.394,5   &954,9  \\
SJC3b    &45.381,0  &1.433,6    &44.164,0   &1.257,6    &42.872,4   &333,9 \\
SJC4a    &73.892,6  &7.058,7    &69.357,4   &884,2  &65.340,5   &1.219,2  \\
SJC4b    &62.457,6  &5.679,8    &56.585,4   &640,5  &55.867,4   &715,6\\
SJC5     &102.325,8    &1.894,5    &109.244,7  &9.669,6    &91.377,1   &2.616,6  \\
AESM\_10    &22.434,58 &5.723,65   &21.166,50  &3.880,00   &19.874,29  &2.410,11 \\
AESM\_20    &19.000,31  &2.825,51   &16.322,42  &1.568,66   &14.684,51  &1.445,29\\
AESM\_30    &16.080,65  &1.764,01   &15.418,38  &1.564,28   &13.392,51  &928,93\\
AECP\_10    &6.594,54   &61,89  &8.130,36   &3.288,99   &6.560,09   &112,85 \\
AECP\_20    &6.318,05   &349,87 &6.702,14   &564,02 &6.107,97   &144,67\\
AECP\_30    &10.319,98  &4.241,99   &7.239,48   &2.448,47   &6.482,20   &1.455,36\\
AENH\_10    &28.441,94  &2.038,19   &29.939,45  &7.388,23   &24.126,65  &503,47\\
AENH\_20    &26.918,98  &1.808,29   &26.577,87  &1.844,80   &23.861,87  &742,50\\
AENH\_30    &25.554,47  &693,58 &28.506,26  &7.758,78   &23.412,79  &561,11\\
AELJ\_10    &14.735,16  &4.248,58   &11.158,81  &1.699,12   &10.782,29  &372,57\\
AELJ\_20    &14.386,39 &5.332,40   &8.883,50   &528,24 &8.543,30   &244,10\\
AELJ\_30    &14.186,13    &5.082,51   &8.491,72   &601,71 &8.138,73   &70,00 \\
AEAL\_10    &12.527,41   &4.291,75   &11.779,82  &330,40 &10.901,17  &68,26 \\
AEAL\_20    &12.438,82  &4.874,71   &10.504,24  &792,56 &9.812,93   &732,16 \\
AEAL\_30    &11.998,70  &3.951,95   &10.980,83  &2.566,85   &9.297,77   &648,16 \\
\bottomrule
\end{tabular}
\begin{tablenotes}
 \item[1]Desvio Padr�o
\end{tablenotes}
\end{center}
\end{threeparttable}


Os tempos computacionais tomados por cada algoritmo para gerar as solu�es apresentadas anteriormente, � apresentado na tabela
\ref{tab:TemposInstancias}.

\begin{table}[htp]
\begin{center}
\caption {\\Tempo (em segundos) para c�lculo das solu��es.\label{tab:TemposInstancias}}
\begin{tabular}{lrrrrrrrrrrrr}
\toprule
Inst�ncia & \multicolumn{1}{c}{Farthest} & \multicolumn{1}{c}{Density} & \multicolumn{1}{c}{H-Means}
          & \multicolumn{1}{c}{J-Means}  & \multicolumn{1}{c}{Random Density} \\
\midrule
%%%%%%%Valores
SJC1  &  0,009  &0,085   &0,002  & 0,177  & 0,092\\
SJC2  &  0,030  &0,393   &0,003  & 3,002  & 0,726\\
SJC3a &  0,036  &1,678   &0,013  &28,983  & 3,568 \\
SJC3b &  0,046  &1,839   &0,010  &50,773  & 3,484 \\
SJC4a &  0,056  &6,396   &0,021  &74,550  &20,727 \\
SJC4b &  0,054  &6,254   &2,981  &199,509 &8,679\\
SJC5  &  0,050  &5,110   &0,026  &49,322  &6,070 \\
AESM\_10 &   0,87   & 108,90  &0,19 &482,74  &  358,33 \\
AESM\_20 &   0,08   & 53,41  &0,16& 530,03 &266,70 \\
AESM\_30 &   0,08   &57,73   &0,15&375,48  &141,06\\
AECP\_10 &   0,07   & 2,06   & 0,02 & 3,04  & 2,36 \\
AECP\_20 &   0,01   & 1,79   & 0,01 & 2,53  & 3,08 \\
AECP\_30 &   0,01   & 1,39   & 0,01 & 0,86 & 2,24 \\
AENH\_10 &   4,14   &458,44   &1,21 &  4.857,36  &  1.407,63\\
AENH\_20 &    0,26  & 461,44  &1,34 &  3.838,05  & 1.572,15\\
AENH\_30 &   0,26   & 361,01  &0,76 &  3.281,53  &  961,22\\
AELJ\_10 &   0,04   & 7,96   & 5,81 & 54,66 & 55,24 \\
AELJ\_20 &   0,03   & 10,57  & 6,18 & 82,39 & 25,52\\
AELJ\_30 &   0,04   & 1,39   &0,04  &53,57  &20,55  \\
AEAL\_10 &  0,18    & 13,04  &0,03  &10,59  &13,66 \\
AEAL\_20 &  0,02    & 6,21   &0,03  &24,00  &9,03 \\
AEAL\_30 &  0,03    &4,40    &0,03  &19,47  &15,70\\
\bottomrule
\end{tabular}
\end{center}
\end{table}


% \begin{table}[htp]
% \begin{center}
% \caption {\\Valores da obtidos das inst�ncias pelos algoritmos construtivos.\label{tab:valoresInstancias}}
% \begin{tabular}{lrrrrrrrrrr}
% \toprule
% Inst�ncia & \multicolumn{2}{c}{Density} & \multicolumn{2}{c}{Farthest} & \multicolumn{2}{c}{H-Means}
%           & \multicolumn{2}{c}{J-Means}  & \multicolumn{2}{c}{Random Density} \\
% &\multicolumn{1}{c}{Valor}&\multicolumn{1}{c}{T(s)}
% &\multicolumn{1}{c}{Valor}&\multicolumn{1}{c}{T(s)}
% &\multicolumn{1}{c}{Valor}&\multicolumn{1}{c}{T(s)}
% &\multicolumn{1}{c}{Valor}&\multicolumn{1}{c}{T(s)}
% &\multicolumn{1}{c}{Valor}&\multicolumn{1}{c}{T(s)} \\
% \midrule
% %%%%%%%Valores
% SJC1  &18.193,2 &0,085& 25.905,0 & 0,009 &19.169,4 &0,002&20.247,9 &0,177 &99999 &0,1\\
% SJC2  & 34.520,4  &0,393&52.121,3 &0,030&35.179,5&0,003&34.591,2&3,002&99999 &0,1\\
% SJC3a & 47.976,7  &1,678&76.345,3&0,036&49.231,8&0,013&48.140,4& 28,983 &99999 &0,1\\
% SJC3b & 42.501,7  &1,839&69.956,2& 0,050& 42.607,4&0,010&42.329,0&50,773 &99999 &0,1\\
% SJC4a & 65.655,4  &6,396&103.171,0 &0,056&66.763,9&0,021&67.595,4&74,550 &99999 &0,1\\
% SJC4b & 54.077,1  &6,254&92.948,9 &0,054& 57.092,1&2,981&55.412,2&199,51 &99999 &0,1\\
% SJC5 & 97.681,6  &5,110&134.798,0 &0,050& 99.502,9 &0,026&100.117,0&49,322 &99999 &0,1\\
% AESM\_10 & 99999 &0,1&99999 &0,1&99999 &0,1&99999 &0,1 &99999 &0,1\\
% AECS\_10 & 99999 &0,1&99999 &0,1&99999 &0,1&99999 &0,1 &99999 &0,1\\
% AENH\_10 & 99999 &0,1&99999 &0,1&99999 &0,1&99999 &0,1 &99999 &0,1\\
% AESL\_10 & 99999 &0,1&99999 &0,1&99999 &0,1&99999 &0,1 &99999 &0,1\\
% AESM\_20 & 99999 &0,1&99999 &0,1&99999 &0,1&99999 &0,1 &99999 &0,1\\
% AECS\_20 & 99999 &0,1&99999 &0,1&99999 &0,1&99999 &0,1 &99999 &0,1\\
% AENH\_20 & 99999 &0,1&99999 &0,1&99999 &0,1&99999 &0,1 &99999 &0,1\\
% AESL\_20 & 99999 &0,1&99999 &0,1&99999 &0,1&99999 &0,1 &99999 &0,1\\
% AESM\_30 & 99999 &0,1&99999 &0,1&99999 &0,1&99999 &0,1 &99999 &0,1\\
% AECS\_30 & 99999 &0,1&99999 &0,1&99999 &0,1&99999 &0,1 &99999 &0,1\\
% AENH\_30 & 99999 &0,1&99999 &0,1&99999 &0,1&99999 &0,1 &99999 &0,1\\
% AESL\_30 & 99999 &0,1&99999 &0,1&99999 &0,1&99999 &0,1 &99999 &0,1\\
% \bottomrule
% \end{tabular}
% \end{center}
% \end{table}
Tomando como base os resultados apresentados por Lorena e por negreiros para as inst�ncias SJC, escolhemos os melhores resultados, que foram os apresentados por Lorena (CITE LORENA), apresentamos a tabela \ref{tab:CompResultados} onde temos os melhores resultado obtidos pelos trabalhos e por os implementados por n�s. Como pode ser visto, Lorena possui os melhores resultados, pos isso servir�o de base para calcular a porcentagem que as demais solu��es se encontram acima de suas solu��es.


\begin{threeparttable}[ht]
\begin{center}
\caption {\\Compara��o dos resultados com a literatura.\label{tab:CompResultados}}
\begin{tabular}{lrrrrrrrrrrr} \hline

 &  \multicolumn{2}{c}{N\tnote{1}} &   \multicolumn{2}{c}{L\tnote{2}}   &  \multicolumn{2}{c}{TCC\tnote{3}}  &   \multicolumn{2}{c}{N(\%)}     &   \multicolumn{2}{c}{TCC(\%)}\\
   & \multicolumn{1}{c}{V\tnote{a}} &  \multicolumn{1}{c}{T\tnote{b}}& \multicolumn{1}{c}{V} &  \multicolumn{1}{c}{T}& \multicolumn{1}{c}{V} &  \multicolumn{1}{c}{T}& \multicolumn{1}{c}{V(\%)} &  \multicolumn{1}{c}{T(\%)}& \multicolumn{1}{c}{V(\%)} &  \multicolumn{1}{c}{T(\%)} \\ \hline
SJC1    &17288,9  &  0,02   & 17252,1  &   68,62  & 17693,9   &0,09   & 0,21 &  0,029 & 2,56  & 0,13 \\
SJC2    &33370,2 &   0,02   & 33223,6   &  2083,92 & 34147,2  &0,72   & 0,44 &  0,001 & 2,78  & 0,03 \\
SJC3a   &45335,1  &  0,08  & 45313,4   &  2604,92 & 45907,9  & 3,56  &  0,05 &  0,003 & 1,31 &  0,13 \\
SJC3b   &0        &        0 &  40634,9  &  867,68 & 42151,3  &3,48   &    |  &         |&  3,73 &  0,40 \\
SJC4a   &62026,9  &  0,08  & 61842,4   &   27717,11&  63602,9 &20,72 & 0,30 &  0,000 & 2,85 &  0,07 \\
SJC4b   &0         &       0 &  52396,5   &  4649,47 &54077,1  &6,25   &       |&        |&  3,21  & 0,13 \\
 \hline
\end{tabular}
\begin{tablenotes}
 \item[1]Negreiros \cite{negreiros2006capacitated}
 \item[2]Lorena \cite{negreiros2006capacitated}
 \item[3]Algoritmos implementados
\end{tablenotes}
\begin{tablenotes}[para]
 \item[]
 \item[a]Valor da fun��o objetivo,
 \item[b]Tempo gasto pelo algoritmo em segundos.
\end{tablenotes}
\end{center}

\end{threeparttable}


